\documentclass[12pt,a4paper]{article}

% Packages
\usepackage[margin=1in]{geometry} 
\usepackage{setspace} 
\usepackage{titlesec} 
\usepackage{amsmath}
\usepackage[numbers]{natbib}

% Title formatting
\titleformat{\section}[block]{\large\bfseries}{\thesection}{1em}{}
\usepackage{fancyhdr}
\pagestyle{plain}

\begin{document}

% Title Page
\begin{titlepage}
    \centering
    \vspace*{2cm}
    \Huge\textbf{Progress Report} \\
    \vspace{2cm}
    \large{Understanding the Quantum Behaviour of a Two Level System Coupled to a Quantum Harmonic Oscillator} \\
    \vspace{1cm}
    Submitted by: Rowan Adeya \\
    Date: \today \\
    \vspace{1cm}
    \small{\textit{Supervisor: Dr Alexandra Olaya-Castro}} \\
    \small{\textit{Co-Supervisor: Chawntell Kulkarni}} \\
    

    \vfill
    \normalsize
    University College London
\end{titlepage}

% Main content
\setcounter{page}{1}

%%%%%%%%%%%%%%%%%%%%%%%%%%%%%%%%
\section{Introduction}
The aim of this project is to understand the quantum behaviour of a Two-Level System (TLS) coupled to a Quantum Harmonic Oscillator (QHO). In 1963, Edwin Jaynes and Fred Cummings set out to model a single atom interacting with the quantized electromagnetic field within an optical cavity\cite{C_JC_1963} using the TLS-HO model. As such, they formulated the Jaynes-Cummings Model (JCM), which uses the Rotating-Wave Approximation to simplify the analysis of their system. Later, Herbert Walther experimentally verified\cite{C_JC_verification} the JCM, proving its importance within the field of Quantum Optics. Presently, the JCM has been extended to various scientific fields and is thus ubiquitous. It can describe atoms in an optical cavity \cite{H_JC_friction}, superconducting qubits strongly coupled to an electromagnetic environment\cite{C_superconducting_qubit} and, in the field of chemical physics, electron and proton transfer in solvent environments \cite{C_superconducting_qubit}, amongst many other systems.\\
\\
To understand the quantum behaviour of a TLS-HO system, we begin with the JCM Hamiltonian. We shall look at different forms of interaction between the system, using both closed and open quantum approaches. Furthermore, we shall study its entanglement and coherence properties to gain further insight into the system's behaviours. Entanglement is a measure of the mixedness of a quantum system, indicating how much the system's state deviates from being a separable (product) state\cite{E_WC_Definition} and how much quantum correlation exists between subsystems. Quantifying the entanglement of our system gives us non-classical insight into our system, necessitating the study of this phenomenon. Coherence\cite{C_colloquium}, another quantum phenomenon related to entanglement, tells us about the phase relationship between different states of the total system; a loss of coherence (decoherence) represents a system's transition to classical regime. Studying coherence gives valuable insight into our system's quantum behaviour and optimisation, as it reveals how well the system preserves its states.\\
\\
This progress report summarises the work done so far on both the Research Essay and the Final Project, and outlines future aims and goals. We begin by looking at preliminary investigations. We then look at the progress made on the literature review. Finally, we discuss the future aims and goals.


%%%%%%%%%%%%%%%%%%%%%%%%%%%%%%%%
{\section{Preliminary Investigations}}
The majority of the work done so far, as detailed in the Project Outline, is preliminary investigations and basic closed quantum dynamics of the model. We firstly present the Hamiltonian describing the TLS-HO system. We then diagonalise the Hamiltonian, which will be useful when we calculate the time-evolved state using the unitary evolution operator. From here, we calculate pure density matrix and its subsystems. Finally, we use these subsystems to quantify entanglement measures. 
\\
\subsection{\normalsize{Diagonalizing the Hamiltonian of the TLS-QHO System}}

We first define the Jaynes-Cummings\cite{H_JC_friction} Hamiltonian\cite{C_spin-boson} $\hat{H}$, which describes a TLS interacting with a QHO; it will later be of use when analysing the time evolution of the system. 

\begin{equation}
    \hat{H} = \frac{\hbar\omega_1}{2}\hat{\sigma}_z + \hbar\omega_2(a^\dagger a + \frac{1}{2}) + \hbar G(a\hat{\sigma}_{eg} + a^\dagger\hat{\sigma}_{ge}),
\end{equation}
\\
where $\omega_1$ is the energy difference between the two levels of the TLS, $\omega_2$ is the frequency of the harmonic oscillator, $\sigma_z$ is the Pauli spin operator, $\sigma_{eg}$ and $\sigma_{ge}$ are the respective excitation and de-excitation operators that act on the TLS, $a^\dagger$ and $a$ are the respective creation and annihilation operators that act on the QHO, and $G$ is the coupling strength constant. The Hamiltonian $\hat{H}$ can be split into three different parts representing the contributions from: \eqref{H_TLS} the TLS; \eqref{H_QHO} the QHO; \eqref{H_INT} the interaction between the TLS and QHO. 

\begin{equation*}
    \hat{H} = \hat{H}_{TLS} + \hat{H}_{QHO} + \hat{H}_{int}\\
\end{equation*}
where 
\begin{align}
    \hat{H}_{TLS}\equiv \frac{\hbar\omega_1}{2}\hat{\sigma}_z\label{H_TLS}\\
    \hat{H}_{QHO}\equiv \hbar\omega_2(a^\dagger a + \frac{1}{2})\label{H_QHO}\\
    \hat{H}_{int} \equiv\hbar G(a\hat{\sigma}_{eg} + a^\dagger\hat{\sigma}_{ge})\label{H_INT}
\end{align}
\\
It is important to note that later on in the project, we will be studying different forms of the interaction Hamiltonian $\hat{H}_{int}$.
We may now look at a general joint state of the TLS-QHO system, which is given by:

\begin{equation}
    |\Psi\rangle = \alpha|g,n+1\rangle + \beta|e,n\rangle \label{joint_state},
\end{equation}
\\
where $g$ and $e$ represent the ground and excited states respectively, and n denotes the number of excitations of the Harmonic Oscillator, $n=0,1,2,3,...$ . To diagonalise $\hat{H}$, we must firstly represent the Hamiltonian as a matrix in the joint basis. The diagonalised Hamiltonian Matrix $D$ is then given by:

\begin{equation}
D = \begin{bmatrix}
\hbar\omega_2(n+1) + k & 0 \\
0 & \hbar\omega_2(n+1) - k
\end{bmatrix},
\end{equation}
\vspace{2cm}
where 
\begin{equation*}
    k = \hbar\sqrt{\frac{\omega_1^2 + \omega_2^2}{4}-\frac{\omega_1\omega_2}{2}+G^2(n+1)}.  
\end{equation*}
\\
\subsection{\normalsize{Time Evolved State Vector}}

Now that we have the diagonalised Hamiltonian, we can proceed to calculate the time-evolved state $|\Psi(t)\rangle$ for a given initial state of $|\Psi(t=0)\rangle = |e,0\rangle$. For this initial state, the number of excitations of the QHO $n=0$ (assuming that there is only one excitation in the joint system at any time), and so our constant $k$ now becomes $k'$:

\begin{equation*}
    k' = \hbar\sqrt{\frac{\omega_1^2 + \omega_2^2}{4}-\frac{\omega_1\omega_2}{2}+G^2}
\end{equation*}
\\
Since our diagonalised Hamiltonian is in the eigenbasis, we must convert the initial state into the eigenbasis. Then we use the evolution operator

\begin{equation}
    \hat{U(t)} = e^{-\frac{iDt}{\hbar}}
\end{equation}

to calculate $|\Psi(t)\rangle$ - this process is made far simpler due to the Hamiltonian being diagonalized, since raising a matrix to a power is trivialised if the matrix is diagonal. Using $|\Psi(t)\rangle = \hat{U(t)}|\Psi(t=0)\rangle$, we obtain the final time-evolved state:

\begin{equation}
    |\Psi(t)\rangle = \begin{bmatrix}
        e^{-i(\omega_2 + \frac{k'}{\hbar})t}(1+(\frac{a+k'}{c})^2)^{-\frac{1}{2}}(\frac{a+k'}{c}) \\
        e^{-i(\omega_2 - \frac{k'}{\hbar})t}(1+(\frac{a-k'}{c})^2)^{-\frac{1}{2}}(\frac{a-k'}{c})
    \end{bmatrix},
\end{equation}
\\
where $a = \hbar(\frac{\omega_1 - \omega_2}{2})$, $c = \hbar G\sqrt{n+1} = \hbar G$ since $n=0$ from our initial state, and the time-evolved state is written in terms of its eigenstates.
\\
\subsection{\normalsize{Time-Evolved Density Matrix}}

In order to calculate entanglement, the easiest way often is to first determine the density matrix for a pure state. In order to find the time-evolved density matrix, we firstly convert $|\Psi(t)\rangle$ back into the original basis (see equation (2)). For the general density matrix, $\rho$, we have:

\begin{equation}
    \rho = \begin{bmatrix}
        \alpha\alpha^* & \alpha\beta^* \\
        \alpha^*\beta & \beta\beta^*
    \end{bmatrix},
\end{equation}

where the first row and first column represents $|g,n+1\rangle$ and $\langle g,n+1|$ respectively, and  and the second row and column represent $|e,n\rangle$ and $\langle e,n|$ respectively. We must then determine the coefficients $\alpha(t)$ and $\beta(t)$ for the time-evolved density matrix. Finally, we perform a partial trace over each respective subsystem (TLS or QHO) to obtain the subsystem density matrices.

\begin{align}
    \rho(t)_{QHO} = \alpha(t)\alpha^*(t)|1\rangle\langle1| + \beta(t)\beta^*(t)|0\rangle\langle0|
    \\
    \rho(t)_{TLS} = \alpha(t)\alpha^*(t)|g\rangle\langle g| + \beta(t)\beta^*(t)|e\rangle\langle e|,
\end{align}

where 

\begin{align*}
    \alpha(t)\alpha^*(t) &=\eta_+^2 + \eta_-^2 + 2\eta_+\eta_-\cos{\frac{k'}{\hbar}t} \\
    \beta(t)\beta^*(t) &= \eta_+^2(\frac{a+k'}{c})^2 + \eta_-^2(\frac{a-k'}{c})^2 + 2\eta_+\eta_-(\frac{a+k'}{c})(\frac{a-k'}{c})\cos{\frac{k'}{\hbar}t} \\
\end{align*}

and 

\begin{equation*}
    \eta_{\pm}= (1+(\frac{a\pm k'}{c})^2)^{-1}(\frac{a\pm k'}{c}).
\end{equation*}
\\
\subsection{\normalsize{Entanglement Quantification}}

From here, we may begin to look at entanglement quantification. Let us consider four different measurements of entanglement: the Von Neumann Entropy of Entanglement\cite{E_VNE_Definition}; the Relative Entropy of Entanglement (REE)\cite{E_REE_generation},\cite{E_REE_VN_applied}; the Entanglement Renyi $\alpha$ Entropy (ER$\alpha$E)\cite{E_ERaE_Definition}; Wooters' Concurrence and Entanglement of Formation \cite{E_WC_Definition}. 

The Von Neumann entropy of Entanglement\cite{E_VNE_Definition} measures the degree of entanglement for a pure bipartite system such as the TLS-QHO system, and depends on either of the subsystem density matrices' eigenvalues. 

\begin{equation}
    S = -\text{Tr}(\rho_{TLS}\ln({\rho_{TLS}})) = -\text{Tr}(\rho_{QHO}\ln({\rho_{QHO}}))
\end{equation}

Furthermore, if the matrix is diagonalised, the Von Neumann entropy reduces to:

\begin{equation}
    S = \sum_i -\lambda_i\ln({\lambda_i})
\end{equation}

Since there is no mixing for either of the two subsystems, their density matrices are diagonal, and the non-zero values are thus the eigenvalues. Here, we make the following approximation\cite{H_JC_friction}: we assume that the TLS and QHO are in resonance, i.e. $\omega_1=\omega_2=\omega$. Using the QHO subsystem, we find that the Von Neumann entropy of entanglement oscillates between $0$ and $\ln{2}$.

The REE measures the minimum "distance" between a given quantum state $\rho$ and all its other disentangled density matrices $\sigma$\cite{E_REE_VN_applied}. In order to calculate REE, it is best done via numerical methods. However, there is a trivial answer for our system\cite{E_REE_generation}: for pure states, REE reduces to the Von Neumann entropy of entanglement.

\begin{equation}
    S(\rho) = \min_{\sigma \in \chi}  D(\rho||\sigma) =-\text{Tr}(\rho_{TLS}\ln({\rho_{TLS}})) = -\text{Tr}(\rho_{QHO}\ln({\rho_{QHO}})),
\end{equation}

where $\chi$ is the set of all the disentangled states.
\\
Wooters states\cite{E_WC_Definition} that the Entanglement of Formation (EoF) measures the resources needed to create a given entangled state. He quantifies it in terms of "concurrence", $C$, which ranges from 0 to 1 (0 being no entanglement present, 1 being a maximally entangled state). 

\begin{equation}
    E(C) = h(\frac{1+\sqrt{1-C^2}}{2}),
\end{equation}

where 

\begin{align*}
    h(x) = -x\log_2{x} - (1-x)\log_2{1-x} \\
    C(\rho) = \text{max}\{0,\lambda_1-\lambda_2-\lambda_3-\lambda_4\}
\end{align*}

for the eigenvalues of the matrix $\rho \tilde{\rho}$, where $\tilde{\rho}$ is the spin-flipped density matrix. Since $E$ ranges from 0 to 1 and concurrence does as well, concurrence is effectively a measure of entanglement. A very important point, however, is that Wooters defines concurrence for a pure density matrix of a two-qubit system; our system only contains one qubit (the TLS), and so concurrence does not apply, and further investigation has to be done on a more generalised formulation of concurrence. 
\\
Finally, ER$\alpha$E is a measure of entanglement which provides more information about the entanglement, since it gives a continuous spectrum parametrised by $\alpha$. For a pure density matrix, it is defined by either one of the subsystems.

\begin{equation}
    R_{\alpha}(\Psi) \equiv \frac{1}{1-\alpha}\log_2{\text{Tr}(\rho_B^\alpha)}
\end{equation}

In the $\alpha \to 1$ limit, the ER$\alpha$E reduces to the Von Neumann entropy, which further motivates the generality of the ER$\alpha$E. It would be interesting to look at different choices of $\alpha$ around $1$ and see how the entanglement changes. 


%%%%%%%%%%%%%%%%%%%%%%%%%%%%%%%%
\section{Literature Review}

The literature review will consist of the following main sections (subject to change): (1) Spin-Oscillator Model; (2) Quantum Resources: Entanglement and Coherence; (3) Applications of Quantum Resources.

For Section 1, we have gathered sources that describe and explain the Spin-Oscillator model (see Section 2 of this report). However, further reading is required on its detailed applications.

For Section 2, significant progress has been made on both entanglement and coherence. We have identified detailed sources on entanglement, while a colloquium\cite{C_colloquium} has been reviewed for coherence, which includes most of the necessary references for this subsection.

For Section 3, sources on the applications of entanglement and coherence have been collected for use.

Reading and source collection for preliminary investigations are closely tied to the literature review. We have compiled a comprehensive list of references, though additional contextual sources are needed to explore the application of the Spin-Oscillator model in greater detail.


%%%%%%%%%%%%%%%%%%%%%%%%%%%%%%%%
\section{Future Aims and Goals}

In the project outline, we set the following goals for the Research Project.

\begin{flushleft}
    \begin{enumerate}
        \item Build an understanding of the joint basis and joint state functions of the compound system.
        \item Investigate the closed unitary quantum dynamics for known system interactions.
        \item Investigate the quantum behaviour (coherence and entanglement) for the system.
        \item Investigate the effects of decoherence on the system using an open system approach, if time permits.
    \end{enumerate}
    
\end{flushleft}

We have thus far completed steps 1-3 for a Spin-Oscillator model of known interaction. Now, we shall repeat steps 1-3 for unknown interaction terms by modifying the interaction Hamiltonian \eqref{H_INT} and repeating our analytical steps. This shall now involve computational methods in Python.

For the literature review, it is imperative that we find many more contextual sources in order to have a comprehensive list of sources. 


\newpage

\bibliographystyle{unsrt} 
\bibliography{References/references.bib} 


\end{document}
