\documentclass[12pt,a4paper]{article}

% Packages
\usepackage[margin=0.75in]{geometry} 
\usepackage{setspace} 
\usepackage{titlesec}
\usepackage[colorlinks=true,linkcolor=black,citecolor=red,urlcolor=red]{hyperref}
\usepackage{amsmath}
\usepackage[numbers,link]{natbib}


% Title formatting
\titleformat{\section}[block]{\large\bfseries}{\thesection}{1em}{}
\usepackage{fancyhdr}
\pagestyle{plain}


\begin{document}

% Title Page
\begin{titlepage}
    \centering
    \vspace*{2cm}
    \Huge\textbf{Quantum Resources in Two-Level Systems Coupled to Quantum Harmonic Oscillators} \\
    \vspace{1cm}
    \small{Submitted by: Rowan Adeya \\
    Date: \today} \\
    \vspace{1cm}
    \small{\textit{Supervisor: Dr Alexandra Olaya-Castro}} \\
    \small{\textit{Co-Supervisor: Chawntell Kulkarni}} \\
    

    \vfill
    \normalsize
    University College London
\end{titlepage}

\tableofcontents

\newpage

\section{Introduction} \label{intro}
% Intro to JCM and light description
The quantum description of light-matter interaction can be modelled by a Two-Level System (TLS) interacting with a Quantum Harmonic Oscillator (QHO). In 1963, Edwin Jaynes and Fred Cummings set out to model a single atom interacting with one mode of a quantized electromagnetic field within an optical cavity, cavity Quantum Electrodynamics (CQED), using the TLS-QHO model under the Rotating Wave Approximation (RWA) \cite{Context1963-JC_Original}. As such, they formulated the Jaynes-Cummings Model (JCM), which remains one of the simplest and most important models of light-matter interaction. Later in 1993, Herbert Walther experimentally verified the JCM, proving its importance within the field of CQED \cite{Context1993-JC_Verification}. 
% Applications of JCM 

Presently the JCM, and thus the TLS-QHO model, has been extended to other various areas of research in Physics, including circuit QED \cite{Context2018-Supercond_qubit}, tunnelling phenomena in photonic crystals \cite{Context2012-Tunneling_photons} and trapped ion physics \cite{Context1992-Trapped_ions}. On the other hand, the simplicity of the model is also limiting. The model fails in certain scenarios, such as in the ultra-strong coupling regime in CQED, in systems with more than one TLS or multiple modes of the QHO, and when the RWA is no longer valid \cite{General2024-JC_overview}. The Quantum Rabi Model provides a solution to the RWA problem by including the terms originally ignored by RWA \cite{General2024-JC_overview}. The Dicke Model generalises both the JCM and Rabi Model by allowing for more than one TLS or/and QHO, and by not employing the RWA \cite{General2024-JC_overview}. As we shall see, however, these models are far more complex and usually require computational solutions, whereas the JCM is solvable analytically. 
% Intro and importance of QResources

While more complex models exist, the analytical solvability of the JCM makes it a valuable tool for gaining further insights into light-matter interaction. In particular, understanding how quantum resources manifest within this model provides a foundation for exploring more advanced quantum phenomena. This leads to the study of two key quantum resources: Entanglement and Coherence. Entanglement is a basis-independent measure of how much quantum correlation exists between subsystems of a composite system, indicating how much the system's state deviates from being a separable (product) state \cite{Entanglement2009-Definition}. In a similar vein, coherence is a basis-\textit{dependent} measure, describing the ability of a system to maintain its quantum superposition, and quantifying how well quantum states preserve phase relationships between different basis states \cite{Coherence2017-Colloquium}. While entanglement captures non-separability between subsystems, coherence measures the quantum superposition within a single system or subsystem. Both measures play essential roles in quantum communication and quantum computation, amongst other fields.
% QResources in TLS-QHO etc. 

As quantum technologies progress, achieving precise control over quantum states becomes increasingly important for practical applications. In the context of the TLS-QHO model, entanglement and coherence are key resources that enable both theoretical exploration and experimental control of the system. Entanglement in the TLS-QHO naturally arises from the coupling of the TLS and QHO \cite{Entanglement2009-REE_VNapplied}. Coherence, on the other hand, directly impacts the stability and controllability of the system \cite{Coherence2020-JCMapplied}.  By harnessing these quantum resources, the TLS-QHO model provides a versatile framework for studying and developing future quantum technologies. Given its well-established analytical techniques, the JCM provides an ideal foundation for investigating Quantum Resources in the general TLS-QHO model.  \\
\\
In this review, we investigate the toy-model of the TLS-QHO system through the lens of the JCM, the two Quantum resources of Entanglement and Coherence, and how we may use these resources in our TLS-QHO system. We begin by discussing the history, theoretical aspects, applications, and limitations of the TLS-QHO model in section \ref{sec_TLS-QHO}. We then explore the theoretical applications of entanglement and coherence in section \ref{sec_QRes}. We we examine how Quantum Resources can be applied to the TLS-QHO model in section \ref{sec_QRes_app}, and conclude our findings in section \ref{sec_Conclusion}.

\newpage

\section{TLS-QHO Model} \label{sec_TLS-QHO}

\subsection{Introduction and Historical Background}

% Brief history of Quantum mechanics
In 1901, Max Planck introduced the idea of energy quantisation in `packets' of $h\nu$ to explain the energy density of Blackbody Radiation \cite{Context1901-Planck}. In 1905, Albert Einstein famously expanded on this idea in the seminal set of papers named \textit{Annus Mirabilis} \cite{Context1905-Einstein}. In the context of the Photoelectric Effect, Einstein proposed that light is composed of discrete packets of energy, which he labelled as `Quanta'. From this point on, Quantum Theory was beginning to gain traction, and in 1924 Louis de Broglie introduced the idea that particles could exhibit both wave-like and matter-like properties \cite{Context1924-DeBroglie}. This was a pivotal moment in the development of Quantum Theory. Theoretical research was greatly accelerated to develop the field with key figures, such as Werner Heisenberg and Erwin Schrödinger, pushing the boundaries and testing untouched waters. \\
\\
% Rabi TLS-QHO leading to JC
Isidor Rabi was one of the first physicists to develop the theoretical framework of Quantum light-matter interaction by looking at a single Two-Level System (TLS) coupled to a Quantum Harmonic Oscillator (QHO) \cite{Context1936-Rabi}. A TLS refers to any quantum system with two distinct eigenstates; in the context of the model of light-matter interaction, it refers to an atom with ground and excited energy levels. The QHO refers to a system whose energy levels are quantised; in this context, it refers to a quantised electromagnetic field. As we shall later discuss, Rabi's model, whilst a fairly general model of light-matter, is difficult to analytically solve and often requires computational methods. It remained one of the only reputable models of the TLS-QHO until nearly 30 years later, when in 1963 Edwin Thompson Jaynes and Fred Cummings developed the famous Jaynes-Cummings Model (JCM) of the TLS-QHO \cite{Context1963-JC_Original} whilst studying the field of Quantum Optics.
% JCM and theoretical background
The two physicists set out to clarify the relationship between quantum theory of radiation and the semi-classical theory, and to apply these results to a study of the amplitude and frequency stability of a molecular beam maser. In doing so, they produced a Hamiltonian $H$ (which describes the unitary dynamics of the TLS-QHO model, according to the Schrödinger Equation) composed of three parts \cite{Hamiltonian2012-JC_Friction}: (i) The TLS contribution $H_{TLS}$; (ii) the QHO  contribution $H_{QHO}$; and (iii) The interaction contribution $H_{int}$, which describes the coupling between between the TLS and QHO. The Jaynes-Cummings (JC) Hamiltonian may be written as

\begin{equation}
    \hat{H} = \frac{\hbar\omega_1}{2}\hat{\sigma}_z + \hbar\omega_2(a^\dagger a + \frac{1}{2}) + \hbar G(a\hat{\sigma}_{eg} + a^\dagger\hat{\sigma}_{ge}), 
\end{equation} \label{JC_H}
where 
\begin{align*}
    \begin{aligned}
        \hat{H}_{TLS} &\equiv \frac{\hbar\omega_1}{2}\hat{\sigma}_z \\
        \hat{H}_{QHO} &\equiv \hbar\omega_2(a^\dagger a + \frac{1}{2}) \\
        \hat{H}_{int} &\equiv \hbar G(a\hat{\sigma}_{eg} + a^\dagger\hat{\sigma}_{ge})
    \end{aligned}
\end{align*}

In equation \eqref{JC_H}, $\omega_1$ is the energy difference between the two levels of the TLS, $\omega_2$ is the frequency of the harmonic oscillator, $\sigma_z$ is the Pauli spin operator, $\sigma_{eg}$ and $\sigma_{ge}$ are the respective excitation and de-excitation operators that act on the TLS, $a^\dagger$ and $a$ are the respective creation and annihilation operators that act on the QHO, and $G$ is the coupling strength constant. 

One of the most striking predictions that Jaynes and Cummings made was the presence of Rabi oscillations, which is the periodic exchange of energy between the QHO and the TLS, causing the TLS to oscillate between its two states; this oscillation frequency is known as the Rabi Frequency \cite{Context_-Rabi_oscillations}. 

At the time, however, the technological capabilities required to experimentally verify the JCM and its predictions were not yet available. Despite this, the model garnered support from the physics community, and from the 1960s to the 1980s, theoretical work was done to build upon and extend the foundational framework laid by Jaynes and Cummings \cite{Context1965-TheoreticalJCM, Context1980-TheoreticalJCM, Context1984-TheoreticalJCM}. It wasn't until 1993 that the JCM was experimentally verified by Herbert Walther in the field of cavity Quantum Electrodynamics (cQED) \cite{Context1993-JC_Verification}. Walther injected Rydberg atoms into a superconducting cavity (a `Maser') to study non-classical radiation, mimicking the single TLS, single QHO setup of the JCM. He measured the probability of finding the Rydberg atom in the upper maser level after passing through the cavity, and found that the predictions of the JCM almost exactly matched experimental observations. Furthermore, Rabi oscillations were detected.\\
\\
What began as a quantitative description of the maser in Quantum Optics has now developed into a key model of understanding the intimate relationship between light and matter. Thus from this point onwards, the JCM came to be one of the most successful models of TLS-QHO systems; its success can be attributed to its analytical simplicity and the accuracy of its predictions. This foundational model has found numerous applications in a wide variety of fields, which we shall now study in detail.

\subsection{Applications of the JCM}

\subsubsection{Cavity QED}

The JCM was experimentally verified in the context of cQED, and it continues to be used today to advance this field. Cavity QED (cQED) describes the interaction between atoms and the quantized electromagnetic field inside a cavity, and allows physicists to gain insight into the radiative properties of atoms by modifying the electromagnetic field \cite{General2024-JC_overview}.\\
\\
In 1995, C.K. Law and J. H. Eberly presented a cQED interaction that forces the ground state of the cavity mode to evolve into an arbitrary quantum state \cite{Context1996-CQED_JCM}. They introduce this interaction by proposing a new approach to control the quantum states of a cavity field: manipulate the quantum states of the atom during the interaction process, rather than before the interaction. Notably, they model their system using a unitarily evolved JC Hamiltonian, and extend the model by injecting time-dependency into the coupling constant $g$ in \eqref{JC_H}.\\
\\
In 2012, Yu-Li Dong et al. investigated binary transmission (the process of sending information using two distinct states, typically 0 or 1) in two coupled cavities, each containing a Three-Level Atom \cite{Context2012-CQED-JCM}. They adiabatically eliminate the intermediate state, allowing each cavity to be described by the JCM. They explicitly mention that the JCM provides simple analytical solutions due to the Hamiltonian being exactly diagonalizable, which enables them to focus on the binary transmission itself. In particular, they compare the scheme using JCM analysis to one that uses the Jaynes-Cummings-Hubbard model (JCHM), a model which extends the JCM to multiple TLS, multiple mode QHOs and facilitates the study of many-body quantum systems \cite{Context2007-JCH_1,Context2006-JCH_2,Context2006-JCH_3}. In contrast to the JCM's coupling strength $g$, if we maintain the strong coupling regime, the two-photon coupling of the JCHM $\lambda\approx0.1g$. The quantum transfer of information in the JCHM is mentioned to be $5$ times slower than the JCM due to the weaker coupling in the JCHM. On the other hand, the JCHM allows for a significantly larger propagation distance compared to the JCM, thanks to the long lifetime of metastable states. This extended distance is a result of the slower dynamics in the JCHM, which enables more efficient transfer over longer distances. While the JCM remains a powerful tool for analysing binary transmission in cQED, the JCHM's extension to multiple TLS and longer propagation distances makes it a valuable model for more complex, many-body systems. This paper underscores the versatility of the JCM, which, despite its limitations, continues to play a key role in advancing the study of quantum information transmission in cavity QED.\\
\\
In the present day, the JCM still finds continued use in cQED, as demonstrated by recent studies exploring its applicability in modern quantum frameworks. For example, a 2025 study by Danilo Cius investigates the unitary evolution of the JCM under fractional-time dynamics, extending its formulation using a fractional-time Schrödinger equation \cite{}. Cius highlights how fractional calculus modifies key quantum effects in cQED, such as Rabi oscillations and atomic population inversion, while maintaining a unitary description of the system. By employing a time-dependent Dyson map, the study ensures that the JCM remains analytically solvable even in this modified framework. The paper confirms that entanglement between the atom and field remains strongly dependent on the fractional-time parameter, demonstrating how the JCM continues to serve as a valuable tool in advancing our understanding of quantum coherence and cavity field interactions.

\subsubsection{Circuit QED}

In 2004, Alexandre Blais et al. proposed using superconducting circuits with transmission line resonators to achieve strong coupling in circuit QED, improving qubit lifetimes \cite{Context2004-JC_CQED}. They begin by summarising the field of cQED before proposing their methodology. The authors notably mention the JCM in their review of cQED, stating that the atomic transition frequency  $\omega_1$, cavity resonance frequency $\omega_2$ and coupling strength $g$ from \eqref{JC_H} are key parameters in describing circuit QED systems. Furthermore, during their proposition, their Hamiltonian reduces to the JC Hamiltonian at the charge degeneracy point by neglecting both rapidly oscillating terms and damping. The paper makes it clear that the JCM's strength is in simplifying the system, and allows us to understand the TLS-QHO interaction at its most fundamental level. \\
\\
As with cQED, the JCM again finds relevant use in circuit QED. A 2024 study by Medina-Dozal et al. explores the spectral response of a non-linear Jaynes-Cummings model in the strong-dispersive regime of circuit QED \cite{}. The authors extend the standard JCM by introducing non-linear deformed field operators, leading to an intensity-dependent coupling that affects the system’s spectral features. By deriving analytical expressions for the system’s time-dependent spectral response, the study identifies asymmetries in the spectral structure that deviate from the traditional vacuum Rabi splitting. These spectral modifications provide insights into the effects of non-linearity in superconducting quantum circuits, linking the non-linear JCM to experimentally observed features in circuit QED systems. This work further reinforces the JCM as a foundational model for understanding light-matter interactions in superconducting qubits, particularly in the presence of strong non-linearities and modified coupling mechanisms.

\subsubsection{Alternative Physical Systems}

3 examples of alternative physical systems

\subsection{Limitations and Open Problems}

The two papers addressing key RWA stuff, Dicke, and Rabi Model.




\newpage

\section{Quantum Resources: Entanglement and Coherence} \label{sec_QRes}

Summary of what these quantum resoruces are, why are they relevant? Talk about control of the system.

\subsection{Entanglement}

Define what entanglement is exactly. 

VNE and its importance. 

Renyi

REE

Wooters

\subsection{Coherence}

Define what coherence is. Mention it is still a new field.

Measures of coherence.

Monotones of coherence.


\newpage

\section{Application of Quantum Resources to the TLS-QHO Model} \label{sec_QRes_app}

\newpage

\section{Conclusion} \label{sec_Conclusion}


\newpage

\bibliographystyle{unsrt} 
\bibliography{References/references.bib} 


\end{document}
