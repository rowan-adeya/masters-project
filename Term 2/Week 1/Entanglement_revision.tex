\documentclass[12pt]{article}
\usepackage{amsmath, amssymb, amsthm}
\usepackage{geometry}
\geometry{a4paper, margin=1in}
\usepackage[numbers,super]{natbib}

% Document metadata
\title{Entanglement Measures: Revision and Summary}
\author{Rowan Adeya}
\date{\today}

\begin{document}

\maketitle

\section{Wooters' Concurrence and the Entanglement of Formation (EoF)}

\subsection{Pure States}
For a pure two-qubit system, Wooters defines\cite{E_EoF_and_WC_Derivation} the EoF as:
\begin{equation}
    E(C) = h(\frac{1+\sqrt{1-C^2}}{2}),
\end{equation}

where $C$ is the Concurrence, and ranges between 0 (disentangled) and 1 (maximally entangled). He further states that the EoF reduces to the Von Neumann entropy of entanglement, which we have previously derived to oscillate between $0$ and $\ln(2)$. If we look at a Concurrence value of $1$, we find that $E(C)=\ln(2)$, which matches what we found for the Von Neumann entanglement. \\
\\
This suggests that our system actually behaves like a two-qubit system. This can be explained by the constraint on the QHO to go up to $n=1$ (i.e. the 1-phonon subspace). The QHO then will either be in $|n= 0 \rangle$ or $|n= 1 \rangle$, effectively behaving like a qubit. This is good, especially later on, as many entanglement quantification analyses look at two-qubit systems. 

\subsection{Spin-Flip Operation}
Wooters states that he uses the Spin-flip operation for two purposes:
\begin{enumerate}
    \item To deal with mixed states, and
    \item For expressing the entanglement of a pure state of two qubits.
\end{enumerate}

\subsection{Mixed States}
In his derivation\cite{E_EoF_and_WC_Derivation}, Wooters also considers the mixed states of two-qubit system and defines their entanglement as in "Term 1 Break" pdf. 
\newpage
\section{Relative Entropy of Entanglement (REE)}
The main difficulty with finding the REE lies in its definition for mixed states - that is, the procedure for finding REE is a convex optimisation problem where one has to find the Closest Separable State (CSS)\cite{unread_pdf} to the system by effectively calculating the quasidistance to each state and finding that shortest "distance".

Vedral and Plenio calculate the REE\cite{E_REE_VN_applied} for an open quantum system; we must, however, calculate REE for a mixed state TLS-QHO system using a closed quantum approach (as per our aims of this project). Currently, two papers (see unread references) go over how to find the CSS or/and how to do it numerically. Further investigation is needed.


\newpage
\bibliographystyle{unsrt} 
\bibliography{References/references.bib} 


\end{document}
