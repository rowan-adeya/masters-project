\documentclass{article}
\usepackage[a4paper, margin = 2.4cm]{geometry}
\usepackage{blindtext}
\usepackage{titlesec}
\usepackage{xcolor}
\usepackage[numbers,link]{natbib}
\usepackage[toc,page]{appendix}

%%%%%%%%%%%%%%%%%%%%%%%%%%%%%%%%%%%%%%%%%%%%%%%%%%%%%% TITLE %%%%%%%%%%%%%%%%%%%%%%%%%%%%%%%%%%%%%%%%%%%%%%%%%%%%%%
\title{\textbf{Investigating Two-Level Systems Coupled to Quantum Harmonic Oscillators} \\ Dissertation Structure}
\author{Rowan Adeya}
\date{}
\begin{document}

\maketitle

\newpage

\vspace{\fill}
\begin{abstract}
    Abstract Here. Must be centred within the page. This abstract should define what the system is, motivate the reader, and present findings and conclusions.
\end{abstract}
\vspace{\fill}

\newpage

\tableofcontents

\newpage
%%%%%%%%%%%%%%%%%%%%%%%%%%%%%%%%%%%%%%%%%%%%% INTRODUCTION %%%%%%%%%%%%%%%%%%%%%%%%%%%%%%%%%%%%%%%%%%%%%%%%%%%%%%
\section{Introduction}
This section should do the following:
\begin{enumerate}
    \item Start the introduction by talking about Entanglement and Coherence, as they are the main motivation for studying TLS-QHO models.
    \item  Explain that, in computational systems, we have qubit or qubit-qubit coupled systems, i.e. coupled TLSs. In biological, physical, natural systems, we usually have TLSs coupled to something, and oftentimes this is a QHO.
    \item  Our main question: How do entanglement and coherence manifest in these joint systems? Allows us to exploit resources for quantum information science, etc.
\end{enumerate}
\newpage
%%%%%%%%%%%%%%%%%%%%%%%%%%%%%%%%%%%%%%%%%%%%% TLS-QHO SYS %%%%%%%%%%%%%%%%%%%%%%%%%%%%%%%%%%%%%%%%%%%%%%%%%%%%%%
\section{TLS-QHO Systems}
\subsection{Jaynes-Cummings Model}
\begin{enumerate}
    \item Go over theory of JCM. Technical theoretical section.
\end{enumerate}
\subsection{Exciton-Vibrational Model}
\begin{enumerate}
    \item Go over theory of EVM.
\end{enumerate}
\subsection{Open Quantum Dynamics}
\begin{enumerate}
    \item Talk about the ways in which open system evolution may occur.
    \item Go over Markovian, Lindblad etc.
\end{enumerate}
\newpage
%%%%%%%%%%%%%%%%%%%%%%%%%%%%%%%%%%%%%%%%%%%%% Q RESOURCES %%%%%%%%%%%%%%%%%%%%%%%%%%%%%%%%%%%%%%%%%%%%%%%%%%%%%%
\section{Quantum Resources}
\begin{enumerate}
    \item Why do we need to study quantum resources? how are they helpful?
\end{enumerate}
\subsection{Entanglement}
\begin{enumerate}
    \item Motivate the importance of studying entanglement. 
    \item Key for Q. computation.
\end{enumerate}
\subsubsection{Von Neumann Entropy}
\begin{enumerate}
    \item Define.
    \item Strengths and weaknesses.
    \item Uses.
\end{enumerate}
\subsubsection{Negativity}
\begin{enumerate}
    \item Define.
    \item Strengths and weaknesses.
    \item Uses.
\end{enumerate}
\subsubsection{Other Entanglement Measures}
\begin{enumerate}
    \item Name other entanglement measures of interest.
    \item Wooters' Concurrence
    \item Relative Entropy of Entanglement 
    \item Renyi 
    \item Mention why they aren't being used in this research. 
\end{enumerate}
\subsection{Coherence}
\begin{enumerate}
    \item Motivate the importance of studying entanglement. 
    \item Key for Q. computation.
\end{enumerate}
\subsubsection{Relative Entropy of Coherence}
\begin{enumerate}
    \item Define.
    \item Strengths and weaknesses.
    \item Uses.
\end{enumerate}
\subsubsection{Coherence Norm}
\begin{enumerate}
    \item Define.
    \item Strengths and weaknesses.
    \item Uses.
\end{enumerate}

%%%%%%%%%%%%%%%%%%%%%%%%%%%%%%%%%%%%%%%%%%%%% RESULTS %%%%%%%%%%%%%%%%%%%%%%%%%%%%%%%%%%%%%%%%%%%%%%%%%%%%%%
\section{Results}
\subsection{QuTip and Code}
\begin{enumerate}
    \item Mention use of QuTip package. 
    \item Custom function for calculating Rel. entropy of coherence and negativity -> appendix
    \item Link to github code.
\end{enumerate}

\subsection{Population, Entanglement and Coherence of the Jaynes-Cummings Model}
\subsubsection{Closed Evolution}
\begin{enumerate}
    \item Go over your pen-paper analysis of JCM here.
    \item Define initial conditions and parameters used. 
    \item Analyse all 6 closed evo graphs. 
\end{enumerate}
\subsubsection{Open Evolution}
    \item Define initial conditions and parameters used. 
    \item Analyse all 6 open evo graphs. 
\subsection{Population, Entanglement and Coherence of the Exciton-Vibration Model}
\begin{enumerate}
    \item Define Hamiltonian.
\end{enumerate}
\subsubsection{Closed Evolution}
\begin{enumerate}
    \item Define initial conditions and parameters used. 
    \item Analyse all 6 closed evo graphs. 
\end{enumerate}
\subsubsection{Open Evolution}
    \item Define initial conditions and parameters used. 
    \item Analyse all 6 open evo graphs. 
\subsection{Discussion}
\begin{enumerate}
  \item Compare JCM and ExVib behaviours.
  \item Coherence and entanglement implications.
\end{enumerate}
\newpage

%%%%%%%%%%%%%%%%%%%%%%%%%%%%%%%%%%%%%%%%%%%%% CONCLUSION %%%%%%%%%%%%%%%%%%%%%%%%%%%%%%%%%%%%%%%%%%%%%%%%%%%%%%
\section{Conclusion and Outlook}
\begin{enumerate}
    \item Conclude all findings. What have you garnered from the results. 
    \item Is everything as expected. Were there any deviations?
    \item If so, explain and suggest. 
\end{enumerate}


%%%%%%%%%%%%%%%%%%%%%%%%%%%%%%%%%%%%%%%%%%%%% Appendices %%%%%%%%%%%%%%%%%%%%%%%%%%%%%%%%%%%%%%%%%%%%%%%%%%%%%%
\begin{appendices}
    \section{Code Stuff}
    \begin{enumerate}
        \item We agreed that only code-related stuff like vectorisation, custom functions, etc. should go here. 
        \item Include: a) rel coherence; b) negativity.
    \end{enumerate}
    \section{Vectorisation and the Liouvillian}
    \begin{enumerate}
        \item What is vectorisation?
        \item Explain how Liouvillian is a vectorisation method on Lindblad. 
        \item Explain how it is used in QuTip. 
    \end{enumerate}
\end{appendices}
\newpage

\bibliographystyle{unsrt} 
\bibliography{References/references.bib} 

\end{document}