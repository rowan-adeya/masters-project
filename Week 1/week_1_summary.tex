\documentclass{article}
\usepackage{hyperref}
\usepackage{indentfirst}


\begin{document}

\section*{Week 1 Summary}

\vspace{0.5cm}

\textbf{Organisation.} I decided to organise myself ahead of time so that later on in the project, there is minimal time wastage on administrative issues. I've set up a public GitHub \href{https://github.com/rowan-adeya/masters-project.git}{repository}, containing directories and a simple README.md file. Furthermore, I've linked the repository to my Overleaf account, allowing me to make commits from Overleaf itself.

\vspace{0.5cm}

\textbf{Annotated Bibliography.} In my "Masters Project" Overleaf project, I created a folder which has all the references in a .bib file, as well as a "master" 
.tex file which automatically updates as I place references into the .bib. Whilst this whole process took a long time, I quickly got familiar with how Overleaf works, as prior to this Masters I was only using LaTeX to edit my CVs, and thus only involved editing text. \\ Each "Week n" folder will contain an annotated bibliography of the content covered for that week only, and a summary of the week's work.

\vspace{0.5cm}

\textbf{Tensor product review.} I have covered the basics of tensor product and composite systems, as well as the time-dependent SE and perturbation theory in my Adv Quantum Theory module. I will need to go into more detail on the tensor product and composite systems. 
\vspace{0.5cm}

\textbf{Reflection and aims for Week 2.} Whilst I spent time being productive by organising myself, from now on I will need to make more progress in my learning of Quantum Theory. I will need to discuss what theory to review; so far, I will be looking at the tensor product and composite systems in more detail. I will also aim to submit 2 more annotated bibliographies. 



\end{document}